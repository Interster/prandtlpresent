%%%%%%%%%%%%%%%%%%%%%%%%%%%%%%%%%%%%%%%%%
% Beamer Presentation
% LaTeX Template
% Version 1.0 (10/11/12)
%
% This template has been downloaded from:
% http://www.LaTeXTemplates.com
%
% License:
% CC BY-NC-SA 3.0 (http://creativecommons.org/licenses/by-nc-sa/3.0/)
%
%%%%%%%%%%%%%%%%%%%%%%%%%%%%%%%%%%%%%%%%%

%----------------------------------------------------------------------------------------
%	PACKAGES AND THEMES
%----------------------------------------------------------------------------------------

\documentclass{beamer}

\mode<presentation> {

% The Beamer class comes with a number of default slide themes
% which change the colors and layouts of slides. Below this is a list
% of all the themes, uncomment each in turn to see what they look like.

%\usetheme{default}
%\usetheme{AnnArbor}
%\usetheme{Antibes}
%\usetheme{Bergen}
%\usetheme{Berkeley}
%\usetheme{Berlin}
%\usetheme{Boadilla}
%\usetheme{CambridgeUS}
%\usetheme{Copenhagen}
%\usetheme{Darmstadt}
%\usetheme{Dresden}
%\usetheme{Frankfurt}
%\usetheme{Goettingen}
%\usetheme{Hannover}
%\usetheme{Ilmenau}
%\usetheme{JuanLesPins}
%\usetheme{Luebeck}
\usetheme{Madrid}
%\usetheme{Malmoe}
%\usetheme{Marburg}
%\usetheme{Montpellier}
%\usetheme{PaloAlto}
%\usetheme{Pittsburgh}
%\usetheme{Rochester}
%\usetheme{Singapore}
%\usetheme{Szeged}
%\usetheme{Warsaw}

% As well as themes, the Beamer class has a number of color themes
% for any slide theme. Uncomment each of these in turn to see how it
% changes the colors of your current slide theme.

%\usecolortheme{albatross}
%\usecolortheme{beaver}
%\usecolortheme{beetle}
%\usecolortheme{crane}
%\usecolortheme{dolphin}
%\usecolortheme{dove}
%\usecolortheme{fly}
%\usecolortheme{lily}
%\usecolortheme{orchid}
%\usecolortheme{rose}
%\usecolortheme{seagull}
%\usecolortheme{seahorse}
%\usecolortheme{whale}
%\usecolortheme{wolverine}

%\setbeamertemplate{footline} % To remove the footer line in all slides uncomment this line
%\setbeamertemplate{footline}[page number] % To replace the footer line in all slides with a simple slide count uncomment this line

%\setbeamertemplate{navigation symbols}{} % To remove the navigation symbols from the bottom of all slides uncomment this line
}

\usepackage{graphicx} % Allows including images
\usepackage{booktabs} % Allows the use of \toprule, \midrule and \bottomrule in tables

%----------------------------------------------------------------------------------------
%	TITLE PAGE
%----------------------------------------------------------------------------------------

\title[Bloodlesshound]{The Bloodlesshound Prandtl Wing Project} % The short title appears at the bottom of every slide, the full title is only on the title page

\author{Niel Agenbag} % Your name
\institute[Unaffiliated] % Your institution as it will appear on the bottom of every slide, may be shorthand to save space
{
Unaffiliated \\ % Your institution for the title page
\medskip
\textit{Ludwigprandtlwing@gmail.com} % Your email address
}
\date{\today} % Date, can be changed to a custom date

\begin{document}

\begin{frame}
\titlepage % Print the title page as the first slide
\end{frame}

\begin{frame}
\frametitle{Overview} % Table of contents slide, comment this block out to remove it
\tableofcontents % Throughout your presentation, if you choose to use \section{} and \subsection{} commands, these will automatically be printed on this slide as an overview of your presentation
\end{frame}

%----------------------------------------------------------------------------------------
%	PRESENTATION SLIDES
%----------------------------------------------------------------------------------------

\section{Introduction and Project Goals}

\begin{frame}
\frametitle{Introduction}
\begin{itemize}
\item Bloodlesshound Prandtl wing is an independent fork of a NASA Prandtl wing project
\item See more about it at https://www.nasa.gov/centers/armstrong/news/FactSheets/FS-106-AFRC.html
\item A technical paper (“On Wings of the Minimum Induced Drag: Spanload Implications for Aircraft and Birds,” NASA/TP – 2016-219072) on the research has been published
https://ntrs.nasa.gov/archive/nasa/casi.ntrs.nasa.gov/20160003578.pdf
\item The Bloodlesshound project used this paper as a baseline for further work
\end{itemize}
\end{frame}



\begin{frame}
\frametitle{Introduction Bloodlesshound}

The Prandtl wing is a tailless aircraft concept.  This aircraft is a research vehicle intended to investigate the bell shaped 
lift distribution as suggested by Ludwig Prandtl in his 1933 paper on the efficiency of wings with a specific design mass 
and no span limitation.

The goals of the project are somewhat different from the NASA paper.  The paper focussed on proving the existence of proverse yaw with the Prandtl lift distribution as well as postulating that birds most likely utilize the Prandtl circulation distribution as opposed to the elliptical circulation distribution.  

\end{frame}



\begin{frame}
\frametitle{Project Goals}

The goals are:

\begin{itemize}
\item Investigate whether or not a tailless aircraft designed to have a Prandtl lift distribution will have satisfactory unaugmented (no control system except for the pilot and direct gearing of the control surfaces) handling qualities.
\item Determine the performance characteristics of such an aircraft.  In particular is important to show very favourable L/D characteristics (L/D >= 53 for full scale aircraft) and laminar flow on the wing surfaces, even with reflexed profiles that are common on tailless designs.
\item A secondary goal of the project is to investigate the feasibility of open hardware design using open source tools (as far as possible).  The focus of this goal is to use tools such as Git for product lifecycle management and to manage design projects using mainly text files as the source documents of the design.  Part of the experiment is to test collaboration online with known as well as anonymous collaborators in order to bootstrap mechanical/aeronautical engineering projects.
\end{itemize}

It is hoped that this project will encourage other builders to manufacture similar designs and publish their designs, software and 
data in a similar manner, so as to advance the state of the art of tailless aircraft in particular and engineering in general.

\end{frame}

\begin{frame}
\frametitle{Staged approach}

\begin{itemize}
\item Firstly a small model (1.5metre wing span, polystyrene) will be built to establish manufacturing techniques.  This step has been completed.
\item Secondly a larger 2.25m span model from balsa wood will be manufactured with better performance characteristics.  This is the current step.
\item Lastly larger prototypes have to be designed and built in order to investigate the full scale properties of this type of aircraft.  An intermediate step to the large prototype would be a 27kg all up mass aircraft.  This represents the maximum mass for a legal radio controlled aircraft in South Africa.
\item A full scale aircraft should have a 12 to 15m wing span.  
\end{itemize}

\end{frame}


\begin{frame}
\frametitle{Important notes}

\begin{itemize}
\item It is imperative that a full scale aircraft 
be manufactured since subscale tailless aircraft nearly always have satisfactory handling qualities.  Full scale tailless aircraft 
have to be built in order to investigate the vulnerabilities of tailless aircraft to gusty conditions.
\item The Prandtl wing aircraft is envisaged to be a remotely piloted vehicle in all stages.  This is a risk reduction strategy since the technology of tailless aircraft has inherent risks as evidenced by the literature on the subject.  Even though this is an open hardware project, it is in no way encouraged that a piloted version of this aircraft be built.
\end{itemize}

\end{frame}




\begin{frame}
\frametitle{Project information}

The Bloodlesshound project is an open hardware, open software project.  Information about the project can be found on:  

\begin{itemize}
\item Youtube (search for 'Prandtl Aircraftwing' on google)
\item Instagram (ludwigprandtlwing)
\item Mail the project at: Ludwigprandtlwing@gmail.com 
\item Github (ludwigprandtlwing)
\end{itemize}

\end{frame}




%------------------------------------------------
\section{First Section} % Sections can be created in order to organize your presentation into discrete blocks, all sections and subsections are automatically printed in the table of contents as an overview of the talk
%------------------------------------------------

\subsection{Subsection Example} % A subsection can be created just before a set of slides with a common theme to further break down your presentation into chunks

\begin{frame}
\frametitle{Paragraphs of Text}
Prandtl wing aircraft of the bloodlesshound project:  Second version 2.25metre wing span.



\end{frame}

%------------------------------------------------

\begin{frame}
\frametitle{Bullet Points}
\begin{itemize}
\item Lorem ipsum dolor sit amet, consectetur adipiscing elit
\item Aliquam blandit faucibus nisi, sit amet dapibus enim tempus eu
\item Nulla commodo, erat quis gravida posuere, elit lacus lobortis est, quis porttitor odio mauris at libero
\item Nam cursus est eget velit posuere pellentesque
\item Vestibulum faucibus velit a augue condimentum quis convallis nulla gravida
\end{itemize}
\end{frame}

%------------------------------------------------

\begin{frame}
\frametitle{Blocks of Highlighted Text}
\begin{block}{Block 1}
Lorem ipsum dolor sit amet, consectetur adipiscing elit. Integer lectus nisl, ultricies in feugiat rutrum, porttitor sit amet augue. Aliquam ut tortor mauris. Sed volutpat ante purus, quis accumsan dolor.
\end{block}

\begin{block}{Block 2}
Pellentesque sed tellus purus. Class aptent taciti sociosqu ad litora torquent per conubia nostra, per inceptos himenaeos. Vestibulum quis magna at risus dictum tempor eu vitae velit.
\end{block}

\begin{block}{Block 3}
Suspendisse tincidunt sagittis gravida. Curabitur condimentum, enim sed venenatis rutrum, ipsum neque consectetur orci, sed blandit justo nisi ac lacus.
\end{block}
\end{frame}

%------------------------------------------------

\begin{frame}
\frametitle{Multiple Columns}
\begin{columns}[c] % The "c" option specifies centered vertical alignment while the "t" option is used for top vertical alignment

\column{.45\textwidth} % Left column and width
\textbf{Heading}
\begin{enumerate}
\item Statement
\item Explanation
\item Example
\end{enumerate}

\column{.5\textwidth} % Right column and width
Lorem ipsum dolor sit amet, consectetur adipiscing elit. Integer lectus nisl, ultricies in feugiat rutrum, porttitor sit amet augue. Aliquam ut tortor mauris. Sed volutpat ante purus, quis accumsan dolor.

\end{columns}
\end{frame}

%------------------------------------------------
\section{Second Section}
%------------------------------------------------

\begin{frame}
\frametitle{Table}
\begin{table}
\begin{tabular}{l l l}
\toprule
\textbf{Treatments} & \textbf{Response 1} & \textbf{Response 2}\\
\midrule
Treatment 1 & 0.0003262 & 0.562 \\
Treatment 2 & 0.0015681 & 0.910 \\
Treatment 3 & 0.0009271 & 0.296 \\
\bottomrule
\end{tabular}
\caption{Table caption}
\end{table}
\end{frame}

%------------------------------------------------

\begin{frame}
\frametitle{Theorem}
\begin{theorem}[Mass--energy equivalence]
$E = mc^2$
\end{theorem}
\end{frame}

%------------------------------------------------

\begin{frame}[fragile] % Need to use the fragile option when verbatim is used in the slide
\frametitle{Verbatim}
\begin{example}[Theorem Slide Code]
\begin{verbatim}
\begin{frame}
\frametitle{Theorem}
\begin{theorem}[Mass--energy equivalence]
$E = mc^2$
\end{theorem}
\end{frame}\end{verbatim}
\end{example}
\end{frame}

%------------------------------------------------

\begin{frame}
\frametitle{Figure}
Uncomment the code on this slide to include your own image from the same directory as the template .TeX file.
%\begin{figure}
%\includegraphics[width=0.8\linewidth]{test}
%\end{figure}
\end{frame}

%------------------------------------------------

\begin{frame}[fragile] % Need to use the fragile option when verbatim is used in the slide
\frametitle{Citation}
An example of the \verb|\cite| command to cite within the presentation:\\~

This statement requires citation \cite{p1}.
\end{frame}

%------------------------------------------------

\begin{frame}
\frametitle{References}
\footnotesize{
\begin{thebibliography}{99} % Beamer does not support BibTeX so references must be inserted manually as below
\bibitem[Smith, 2012]{p1} John Smith (2012)
\newblock Title of the publication
\newblock \emph{Journal Name} 12(3), 45 -- 678.
\end{thebibliography}
}
\end{frame}

%------------------------------------------------

\begin{frame}
\Huge{\centerline{The End}}
\end{frame}

%----------------------------------------------------------------------------------------

\end{document} 